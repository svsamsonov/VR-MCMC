%Indicatrice et parenthésages
\newcommand\parenth[1]{\left(#1\right)}
\newcommand\po{\left(}
\newcommand\pf{\right)}
\newcommand\crochet[1]{\left[#1\right]}
\newcommand\accol[1]{\left\{#1\right\}}

%Intervalles à la francaise
%\newcommand\intoo[2]{\left]#1,#2\right[}
%\newcommand\intof[2]{\left]#1,#2\right]}
%\newcommand\intfo[2]{\left[#1,#2\right[}
%\newcommand\intff[2]{\left[#1,#2\right]}

%Intervalles à l'anglaise
\newcommand\intoo[2]{\left(#1,#2\right)}
\newcommand\intof[2]{\left(#1,#2\right]}
\newcommand\intfo[2]{\left[#1,#2\right)}
\newcommand\intff[2]{\left[#1,#2\right]}


\newcommand\indic[1]{\mathds{1}_{#1}}
\newcommand\card[1]{\left\lvert#1\right\rvert}
\newcommand\sumforce[2]{\underset{#1}{\overset{#2}{\sum}}}
\newcommand\prodforce[2]{\underset{#1}{\overset{#2}{\prod}}}
\newcommand\unionforce[2]{\underset{#1}{\overset{#2}{\bigcup}}}
\newcommand\interforce[2]{\underset{#1}{\overset{#2}{\bigcap}}}
\newcommand\supforce[2]{\underset{#1}{\sup}\left(#2\right)}
\newcommand\infforce[2]{\underset{#1}{\inf}\left(#2\right)}
\newcommand\limforce[2]{\underset{#1}{\lim}\left(#2\right)}
\newcommand\argmin[2]{\underset{#1}{\text{argmin}}\left\{#2\right\}}
\newcommand\argmax[2]{\underset{#1}{argmax}\left(#2\right)}
%\DeclareMathOperator*{\argmin}{argmin}
%\DeclareMathOperator*{\argmin}{arg\,min}


%Produit scalaire, valeur absolue, partie entière et norme
\newcommand\ps[2]{\left<#1\middle|#2\right>}
\newcommand\prodscal[2]{\left\langle#1\middle|#2\right\rangle}
\newcommand\abs[1]{\left\lvert#1\right\rvert}
%\newcommand\norm[2]{\left\lvert\left\lvert{#2}\right\rvert\right\rvert_{#1}}
%la commande ci-dessus fait des barres verticales trop espacées pour la norme : celle ci-dessous est bien plus adaptée
\newcommand\norm[2]{\left\|{#2}\right\|_{#1}}
\newcommand\pent[1]{\left\lfloor{#1}\right\rfloor}
\newcommand\pentsup[1]{\left\lceil{#1}\right\rceil}


\newcommand{\eqsp}{\;}
\newcommand{\eqspp}{\ \ }
\newcommand{\eqsppeq}{\quad \quad }


%Raccourcis de lettres stylisées
\newcommand\drm{\mathrm{d}}
\newcommand\erm{\mathrm{e}}



\newcommand{\Pbf}{\mathbf{P}}
\newcommand{\Qbf}{\mathbf{Q}}
\newcommand{\Ebf}{\mathbf{E}}


\newcommand{\bB}{\mathbb{B}}
\newcommand{\Dbb}{\mathbb{D}}
\newcommand{\Pbb}{\mathbb{P}}
\newcommand{\Ebb}{\mathbb{E}}
\newcommand{\Nbb}{\mathbb{N}}
\newcommand{\Zbb}{\mathbb{Z}}
\newcommand{\R}{\mathbb{R}}



\newcommand{\Lcal}{\mathcal{L}}
\newcommand{\cB}{\mathsf{B}}
\newcommand{\cC}{\mathsf{C}}
\newcommand{\cF}{\mathscr{F}}
\newcommand{\cX}{\mathscr{X}}

%Ensembles courants\\
\newcommand\set[2]{\left\lbrace #1 \, : \, #2 \right\rbrace}
\def\Rbb{\mathbb{R}}

\newcommand{\sX}{\mathsf{X}}

%Signes probabilistes
%%%%Convergence
%\newcommand\CV[2]{\overset{\text{\tiny{#1}}}{\underset{\text{\tiny{#2}}}{\longrightarrow}}}
\newcommand\CV[2]{\xrightarrow[\text{\tiny{#2}}]{\text{\tiny{#1}}}}
% '$\xrightarrow[\text{au dessus}]{\text{en dessous}}$' permet de générer des flèches dont la longueur dépend de la longueur du texte qui est placé au dessus ou en dessous
\newcommand\iid{i.i.d. }
\newcommand\as{a.s. }
\newcommand\ie{i.e. }
\newcommand\wrt{w.r.t. }
\def\cdf{c.d.f.}
\def\pdf{p.d.f.}


%%%%Probabilité ``normales''
\newcommand\prob[1]{\mathbb{P}\left(#1\right)}
\newcommand\probpar[2]{\mathbb{P}^{#1}\left(#2\right)}
\newcommand\probcond[2]{\mathbb{P}\left(#1 \middle| #2\right)}
\newcommand\probcondpar[3]{\mathbb{P}^{#1}\left(#2 \middle| #3\right)}
\newcommand\esp[1]{\mathbb{E}\left[#1\right]}
\newcommand\esppar[2]{\mathbb{E}^{#1}\left[#2\right]}
\newcommand\espcond[2]{\mathbb{E}\left[#1 \middle| #2\right]}
\newcommand\espcondpar[3]{\mathbb{E}^{#1}\left[#2 \middle| #3\right]}
\newcommand\var[1]{\mathbb{V}\mathrm{ar}\left(#1\right)}
%%%%Probabilité quenched
\newcommand\probquenched[2]{\mathbf{P}_{#1}\left(#2\right)}
\newcommand\probquenchedpar[3]{\mathbf{P}_{#1}^{#2}\left(#3\right)}
\newcommand\probquenchedcond[3]{\mathbf{P}_{#1}\left(#2 \middle| #3\right)}
\newcommand\probquenchedcondpar[4]{\mathbf{P}_{#1}^{#2}\left(#3 \middle| #4\right)}
\newcommand\espquenched[2]{\mathbf{E}_{#1}\left[#2\right]}
\newcommand\espquenchedpar[3]{\mathbf{E}_{#1}^{#2}\left[#3\right]}
\newcommand\espquenchedcond[3]{\mathbf{E}_{#1}\left[#2 \middle| #3\right]}
\newcommand\espquenchedcondpar[4]{\mathbf{E}_{#1}^{#2}\left[#3 \middle| #4\right]}
%%%%Probabilité annealed
\newcommand\probannealed[2]{\mathbf{P}_{#1}\left(#2\right)}
\newcommand\probannealedpar[3]{\mathbf{P}_{#1}^{#2}\left(#3\right)}
\newcommand\probannealedcond[3]{\mathbf{P}_{#1}\left(#2 \middle| #3\right)}
\newcommand\probannealedcondpar[4]{\mathbf{P}_{#1}^{#2}\left(#3 \middle| #4\right)}
\newcommand\espannealed[2]{\mathbf{E}_{#1}\left[#2\right]}
\newcommand\espannealedpar[3]{\mathbf{E}_{#1}^{#2}\left[#3\right]}
\newcommand\espannealedcond[3]{\mathbf{E}_{#1}\left[#2 \middle| #3\right]}
\newcommand\espannealedcondpar[4]{\mathbf{E}_{#1}^{#2}\left[#3 \middle| #4\right]}
%\newcommand\esptilde[1]{\tilde{\mathbb{E}}\left[#1\right]}
%\newcommand\espsigma[1]{{\mathbb{E}_\sigma}\left[#1\right]}


%Definition d'un nouvel environnement
%%\newaliascnt{notation}{theorem}
%\newtheorem{notation}[notation]{Notation}
%%\aliascntresetthe{notation}
%\crefname{notation}{notation}{notations}
%\Crefname{Notation}{Notation}{Notations}

% Definitions propres à ce documents
\def\prior{\pi}
\newcommand\betaker[2]{\mathbb{B}^{#1,#2}}
\def\param{\vartheta}
\def\Param{\Theta}
\def\Ker{K}
\def\suppprior{\mathrm{supp}(\Pi)}



% noms des hypotheses
\newcommand{\hyptag}[1]{\tag{\ensuremath{\mathbf{#1}}}} % nom d'hypothese

\def\tilde{\widetilde} 
\def\rme{\mathrm{e}}
\def\rmd{\mathrm{d}}
\def\Xsigma{\cX}
\def\Xset{\sX}
\newcommandx{\indi}[2][1=]{\mathbbm{1}^{#1}_{#2}}
\newcommand{\indiacc}[1]{\mathbbm{1}_{\{#1\}}}
\newcommand{\indin}[1]{\mathbbm{1}\left\{#1\right\}}

\newcommandx\sequence[3][2=,3=]
{\ifthenelse{\equal{#3}{}}{\ensuremath{\{ #1_{#2}\}}}{\ensuremath{\{ #1_{#2}, \eqsp #2 \in #3 \}}}}
\newcommandx{\sequencen}[2][2=n\in\N]{\ensuremath{\{ #1, \eqsp #2 \}}}
\newcommandx\sequenceDouble[4][3=,4=]
{\ifthenelse{\equal{#3}{}}{\ensuremath{\{ (#1_{#3},#2_{#3}) \}}}{\ensuremath{\{  (#1_{#3},#2_{#3}), \eqsp #3 \in #4 \}}}}
\newcommandx{\sequencenDouble}[3][3=n\in\N]{\ensuremath{\{ (#1_{n},#2_{n}), \eqsp #3 \}}}

\newcommand{\lrb}[1]{\left[ #1 \right]}
\newcommand{\lrcb}[1]{\left\{ #1 \right\}}
\newcommandx{\chunk}[3][1=0,3=n-1]{{#2}_{#1}^{#3}}
\newcommandx{\tvdist}[3][1=]{\ensuremath{\mathrm{d}^{#1}_{\mathrm{TV}}}(#2,#3)}
\newcommand{\tvdistsym}{\ensuremath{\mathrm{d}_{\mathrm{TV}}}}

\newcommandx{\CPE}[3][1=]{{\mathbb E}_{#1}\left[\left. #2 \, \right| #3 \right]} %%%% esperance conditionnelle
\newcommand{\functionboundeddiff}[2]{\mathbb{BD}(#1,#2)}
\newcommandx\supnorm[2][1=]{| #2 |^{#1}_\infty}
\newcommand{\fracaa}[2]{#1 / #2}
