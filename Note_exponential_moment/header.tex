%\usepackage[notref,notcite]{showkeys}  %  comment out for final version
%\renewcommand*\showkeyslabelformat[1]{\fbox{\normalfont\scriptsize\sffamily#1}}   % for showkeys
\usepackage{natbib}
%\usepackage{enumerate,url,amssymb,  mathrsfs}
\usepackage{graphicx}
\usepackage{xargs}
\usepackage[active]{srcltx}

\usepackage[shortlabels]{enumitem} %label option for enumerate \alph* \Alph* \roman* \Roman* or other... [label=$\bullet$], [label=\alph*]

\usepackage[utf8]{inputenc}
\usepackage[english]{babel}
\usepackage{latexsym}
\usepackage{amssymb}
\usepackage{upgreek}
\usepackage{bbm}
\usepackage{mathrsfs}
\usepackage[norelsize,ruled,vlined]{algorithm2e}
\usepackage{amsmath}
\usepackage[textwidth=4cm,textsize=footnotesize]{todonotes}
%\usepackage[disable]{todonotes}
\usepackage{accents}
\usepackage{dsfont}
\usepackage{hyperref}

\usepackage{aliascnt}
\usepackage{cleveref}
\makeatletter
\newtheorem{theorem}{Theorem}
\crefname{theorem}{theorem}{Theorems}
\Crefname{Theorem}{Theorem}{Theorems}


\newaliascnt{lemma}{theorem}
\newtheorem{lemma}[lemma]{Lemma}
\aliascntresetthe{lemma}
\crefname{lemma}{lemma}{lemmas}
\Crefname{Lemma}{Lemma}{Lemmas}

\newaliascnt{corollary}{theorem}
\newtheorem{corollary}[corollary]{Corollary}
\aliascntresetthe{corollary}
\crefname{corollary}{corollary}{corollaries}
\Crefname{Corollary}{Corollary}{Corollaries}

\newaliascnt{proposition}{theorem}
\newtheorem{proposition}[proposition]{Proposition}
\aliascntresetthe{proposition}
\crefname{proposition}{proposition}{propositions}
\Crefname{Proposition}{Proposition}{Propositions}

\newaliascnt{definition}{theorem}
\newtheorem{definition}[definition]{Definition}
\aliascntresetthe{definition}
\crefname{definition}{definition}{definitions}
\Crefname{Definition}{Definition}{Definitions}


\newaliascnt{definitionProposition}{theorem}
\newtheorem{definitionProposition}[definitionProposition]{Proposition and Definition}
\aliascntresetthe{definitionProposition}
\crefname{Proposition and Definition}{Proposition and Definition}{Proposition and Definition}
\Crefname{Proposition and Definition}{Proposition and Definition}{Proposition and Definition}


\newaliascnt{remark}{theorem}
\newtheorem{remark}[remark]{Remark}
\aliascntresetthe{remark}
\crefname{remark}{remark}{remarks}
\Crefname{Remark}{Remark}{Remarks}


\newtheorem{example}[theorem]{Example}
\crefname{example}{example}{examples}
\Crefname{Example}{Example}{Examples}


\crefname{figure}{figure}{figures}
\Crefname{Figure}{Figure}{Figures}

%%les deux lignes qui suivent concernant "theoremstyle" ont été rajoutées afin de supprimer le point gras automatiquement ajouté à la fin du nom des environnement dans les environnements de type "theorem" : tout cela pour permettre
%\newtheoremstyle{dotless}{}{}{\itshape}{}{\bfseries}{}{0pt}{}
%\theoremstyle{dotless}
\newtheorem{assumption}{\textbf{H}\hspace{-3pt}}
\Crefname{assumption}{\textbf{H}\hspace{-3pt}}{\textbf{H}\hspace{-3pt}}
\crefname{assumption}{\textbf{H}}{\textbf{H}}

\newtheorem{assumptionF}{\textbf{F}\hspace{-3pt}}
\Crefname{assumptionF}{\textbf{F}\hspace{-3pt}}{\textbf{F}\hspace{-3pt}}
\crefname{assumptionF}{\textbf{F}}{\textbf{F}}

\newtheorem{assumptionB}{\textbf{B}\hspace{-3pt}}
\Crefname{assumptionB}{\textbf{B}\hspace{-3pt}}{\textbf{B}\hspace{-3pt}}
\crefname{assumptionB}{\textbf{B}}{\textbf{B}}

\newtheorem{assumptionM}{\textbf{M}\hspace{-3pt}}
\Crefname{assumptionM}{\textbf{M}\hspace{-3pt}}{\textbf{M}\hspace{-3pt}}
\crefname{assumptionM}{\textbf{M}}{\textbf{M}}


\newcommand{\eric}[1]{\todo[color=red!20]{{\bf EM:} #1}}
\newcommand{\erici}[1]{\todo[color=red!20,inline]{{\bf EM:} #1}}
\newcommand{\anici}[1]{\todo[color=green!20,inline]{{\bf AH:} #1}}
%% Les 3 lignes suivantes constitue la technique de Elodie, Matthieu et Sylvain pour obtenir la possibilité d'avoir un environnement hypothèse qui "en quelque sorte dépende d'un paramètre"
%\newcounter{hypH}
%\newenvironment{hypH}{\refstepcounter{hypH}\begin{itemize}
%\item[{\bf H\arabic{hypH}}]}{\end{itemize}}

% Ajout de packages personnels

%\usepackage{stmaryrd}


%% Packages pour la représentation des tableaux
%\usepackage{slashbox}



%%% Local Variables:
%%% mode: latex
%%% TeX-master: "main"
%%% End:
